\section{Research Experience}
\cvSubHeadingListStart

%---
\cvSubheading{University of Science and Technology of Southern Philippines}{May 2023 -- Present}{Researcher $|$ Programmer, \href{https://www.rspotsolutions.com/}{RSPOT}-\href{https://www.ustp.edu.ph/}{USTP} IoT Lab}{Cagayan de Oro, Philippines}

\cvItemListStart

\cvItem{Drafted and secured a ₱5M grant from the Department of Science and Technology, Republic of the Philippines, for a smart traffic solution aimed at reducing congestion by 20\% during peak hours.}

\cvItem{Innovated a custom real-time object detection and tracking algorithm for edge devices, surpassing YOLO with 82\% faster detection time while still maintaining minimal false positives and stable tracking.}

\cvItem{Spearheaded the development of {\hyperref[sec:TrafficEZ]{\underline{TrafficEZ}}}, a \CPP{} computer vision program applying principles of OOP and design patterns. Streamlined the process with a CI/CD pipeline via GitHub Actions and Docker, CTest for unit testing, and automated documentation with Doxygen, saving at least one hour per deployment.}

\cvItem{Conceptualized a campus-scale IoT system for flood disaster management, mentoring four Electronics Engineering undergrads in MING stack, \Csharp{}, and Unity, guiding them to this project's successful execution.}

\cvItemListEnd

%---
\cvSubheading{Nara Institute of Science and Technology}{April 2020 -- March 2023}{Research Assistant, \href{http://imd.naist.jp/}{IMD Lab} $|$ Prof. Hirokazu Kato}{Nara, Japan}

\cvItemListStart

\cvItem{Collaborated on an interdisciplinary team to design AR simulations for surgical training, focusing on the safe, repeatable transfer of tacit knowledge. Work led to a workshop paper published and presented at ISMAR 2022.}

\cvItem{Conducted a systematic review on AR training systems, delineating their distinctions from prevalent AR task support systems, and proposed a categorization of training methods, resulting to a published paper at IEEE TVCG.}

\cvItem{Served as a \Csharp{}/Unity developer creating VR games ({\hyperref[sec:VR Photo Studio]{\underline{VR Photo Studio}} and \href{https://www.dropbox.com/s/zfha1npr9ypn324/OCimd.mp4?dl=0}{\underline{Virtual Castell}}}) for open campus, indirectly boosting the lab's profile and contributing to the enrollment of 16 new students for 2022 and 2023.}

\cvItem{Developed a smartphone interface to complement HoloLens 2's hand gesture input, simplifying AR 3D object manipulation for beginners with buttons, sliders, and virtual joysticks for seamless smartphone\texttt{--}HoloLens interaction.}

\cvItemListEnd

%---
\cvSubheading{University of Trento}{October 2021 -- March 2022}{Visiting Researcher, {MiRo Lab} $|$ Prof. Mariolino De Cecco}{Trento, Italy}

\cvItemListStart

\cvItem{Attained €4.7k ERASMUS{\texttt{\large+}} grant to contribute to the Assisted Unit for Simulating Independent LIving Activities (\href{http://ausilia.tn.it}{\underline{AUSILIA}}) project. This collaboration resulted in one journal publication and multiple conference papers.}

\cvItem{Enhanced therapists' clinical assessments in rehabilitation with AR that provides situated visualizations of patient data, including pressure sensors and skeleton tracking, as evidenced by their positive qualitative feedback.}

\cvItem{Engineered an AR collaboration rehabilitation platform for therapists and patients to interact with simultaneously, manipulating virtual objects via HoloLens 2 and depth cameras, achieving 5mm positional and 1° rotational accuracy.}
\cvItemListEnd

\cvSubHeadingListEnd